%!TEX program = xelatex
% Font Size:
%   10pt, 11pt, 12pt
% Paper Size:
%   a4paper, letterpaper, a5paper, leagalpaper, executivepaper, landscape
% Font Family:
%   roman, sans
\documentclass[12pt, a4paper, roman]{moderncv}

% Style:
%   casual, classic, oldstyle, banking
\moderncvstyle{classic}
% Color:
%   blue, orange, green, red, purple, grey
\moderncvcolor{purple}
% \nopagenumbers{}
% \definecolor{color0}{rgb}{0, 0, 0}
% \definecolor{color1}{RGB}{245, 90, 7}
% \definecolor{color2}{RGB}{39, 40, 34}

% Font specify
\usepackage[UTF8, scheme = plain, heading = false]{ctex}

% Page layout
\usepackage{geometry, graphicx}
\geometry{scale = 0.75}
% \setlength{\hintscolumnwidth}{4cm}           % 如果你希望改变日期栏的宽度
\AtBeginDocument{\settowidth{\hintscolumnwidth}{XXXX 年 -- XXXX 年}}

\AtBeginDocument{\hypersetup{pdfstartview = FitH}}

% Packages
\usepackage{metalogo}
\usepackage{amsmath}
\usepackage{amsfonts}

\providecommand{\CTeX}{\relax}
\renewcommand{\CTeX}{\ensuremath{\mathbb{C}}\TeX}
\usepackage{paralist}


% Self-info
\name{顾}{剑}
% \title{简历}
% \address{街道及门牌号}{邮编及城市}
\email{gujian32@gmail.com}
\phone[mobile]{+86~186~2153~7075}
% \phone[fixed]{+2~(345)~678~901}
% \phone[fax]{+3~(456)~789~012}
% \homepage{home.page}
% \extrainfo{附加信息 (可选项)}
% \photo[<height>][<width-of-frame>]{<file-name>}
% \photo[64pt][0.4pt]{picture}
% Motto
% \quote{}

% 显示索引号;仅用于在简历中使用了引言
%\makeatletter
%\renewcommand*{\bibliographyitemlabel}{\@biblabel{\arabic{enumiv}}}
%\makeatother

% 分类索引
% \usepackage{multibib}
% \newcites{book,misc}{{Books},{Others}}

\begin{document}
\maketitle

\section{教育背景}
\cventry{2009 年 -- 2013 年}{工学学士}{上海交通大学}{}{\textit{电子工程系/信息工程}}{}

% \section{毕业论文}
% \cvitem{题目}{\emph{题目}}
% \cvitem{导师}{导师}
% \cvitem{说明}{\small 论文简介}

\section{工作经历}
\cventry{2019.10 -- 2021.03}{中通快递}{Java工程师}{末端研发部}{C端/营销中台}{ 
  \begin{compactitem}
    \item 中通快递小程序(微信、支付宝、百度)后端开发、公众号、服务窗后端开发,第三方用户数据维护, 同步粉丝数据消息中心
    \item 中通快递小程序末端负责对接交易中心、支付中心, 接入微信免密支付 
    \item 中通快递营销中台用户信息全量同步
    \item 中通快递营销中台优惠券推荐、优惠券冻结
    \item 实践测试驱动开发,随需求的变更进行新增测试用例
  \end{compactitem}
}
\cventry{2017.04 -- 2019.10}{上海中卫医疗健康咨询有限公司}{Java工程师}{}{}{
  \begin{compactitem}
    \item Cool健康App,公众号后端的开发,对接网易云信,短信、智能锁等服务
    \item 拆分项目、集成spring-cloud-kubernetes,进行容器化部署
    \item 环球救援依从性项目开发-使用JWT进行授权和认证
  \end{compactitem}
}
\cventry{2015.04 -2017.04}{上海阿京妈网络科技有限公司}{Java工程师}{}{}{
  \begin{compactitem}
    \item 巴比商城商品模块的设计与开发
    \item 巴比商城门店模块的设计与开发
  \end{compactitem}
}
\cventry{2013.07 -- 2014.05}{上海万达信息股份制有限公司}{Java工程师}{}{}{
  \begin{compactitem}
    \item 上海规土局内网门户网站
    \item 江阴土地拍卖系统
  \end{compactitem}
}

\section{计算机技能}
% \cvdoubleitem{C/C++}{熟悉,曾在老师指导下为同级同学讲课}{Cuda C}{熟悉,曾开发基于 GPU 的高性能大数计算库}

\cvitem{Java} {
  \begin{compactitem}
    \item JCU: 读懂AQS,CMPXCHG (hotspot代码看不懂)
    \item GC: 设定一下GC方法,Metaspace, Heap大小
    \item Collection: 好长时间不看了
  \end{compactitem}
 }
\cvitem{消息中间件}{
  \begin{compactitem}
    \item rabbitmq: 了解几种Exchange区别
    \item kafka:    了解使用方法、分片
  \end{compactitem}
}
\cvitem{分库分表} {
  \begin{compactitem}
    \item shardingsphere-jdbc: 开源框架个人集成过,原理了解程度大概是快速找到RewriteEngine打印生成后的sql日志的水平
    \item sharding-jdbc-dangdang: 
    \item mysql-bin-log: 监听数据更改日志进行业务处理
  \end{compactitem}
}
\cvitem{大数据} {
  \begin{compactitem}
    \item Kafka-Connect: 集成ShardingSphere到Kafka-Connect-Jdbc-Sink中,发现通过ShardingProxy也可以实现
    \item Kafka-Stream, Flink:  读过些书的水平
    \item Sqoop: 看了一下导出mysql的源码,内部使用mysqldump导出
    \item Hive:  阅读几篇tutorial的水平
  \end{compactitem}
}
\cvitem{rxstream} {
  \begin{compactitem}
    \item RxJava:读过一本的书的程度
    \item Reactor:个人使用SpringWebFlux重构并行流业务
  \end{compactitem}
}

\cvitem{Git}{熟悉, Git-Flow}
% \cvitem{MS Office}{熟悉}

\section{外语技能}
% \section{语言技能}
% \cvitemwithcomment{中文}{母语}{}
\cvitemwithcomment{英语}{熟练阅读}{}

\section{备注}
\includegraphics[width=\linewidth]{./images/auth-by-code.png}

% \section{其他 1}
% \cvlistitem{项目 1}
% \cvlistitem{项目 2}
% \cvlistitem{项目 3}

% \renewcommand{\listitemsymbol}{-}             % 改变列表符号

% \section{其他 2}
% \cvlistdoubleitem{项目 1}{项目 4}
% \cvlistdoubleitem{项目 2}{项目 5\cite{book1}}
% \cvlistdoubleitem{项目 3}{}

% % 来自BibTeX文件但不使用multibib包的出版物
% %\renewcommand*{\bibliographyitemlabel}{\@biblabel{\arabic{enumiv}}}% BibTeX的数字标签
% \nocite{*}
% \bibliographystyle{plain}
% \bibliography{publications}                    % 'publications' 是BibTeX文件的文件名

% 来自BibTeX文件并使用multibib包的出版物
%\section{出版物}
%\nocitebook{book1,book2}
%\bibliographystylebook{plain}
%\bibliographybook{publications}               % 'publications' 是BibTeX文件的文件名
%\nocitemisc{misc1,misc2,misc3}
%\bibliographystylemisc{plain}
%\bibliographymisc{publications}               % 'publications' 是BibTeX文件的文件名

\end{document} 